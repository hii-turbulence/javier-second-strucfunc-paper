\documentclass[fleqn,usenatbib, useAMS, a4paper]{mnras}


\usepackage{savesym}
\savesymbol{tablenum}
\usepackage{siunitx}
\restoresymbol{SIX}{tablenum}


% MNRAS is set in Times font. If you don't have this installed (most LaTeX
% installations will e fine) or prefer the old Computer Modern fonts, comment
% out the following line
\usepackage{newtxtext}
\usepackage[varg,varvw,smallerops]{newtxmath}
% Depending on your LaTeX fonts installation, you might get better results with one of these:
%\usepackage{mathptmx}
%\usepackage{txfonts}

% Use vector fonts, so it zooms properly in on-screen viewing software
% Don't change these lines unless you know what you are doing
\usepackage[T1]{fontenc}
\usepackage{ae,aecompl}


%%%%% AUTHORS - PLACE YOUR OWN PACKAGES HERE %%%%%

% Only include extra packages if you really need them. Common packages are:
\usepackage{graphicx}	% Including figure files
\let\Bbbk\relax
\usepackage{amsmath}	% Advanced maths commands
\usepackage{amssymb}	% Extra maths symbols
\usepackage{multicol}
\usepackage{enumerate}          % Better lists
\usepackage{xcolor}

\usepackage[spanish,es-minimal,english]{babel}

%% Package set-up
\usepackage{booktabs}
\usepackage{array}   % for \newcolumntype macro
\newcolumntype{L}{>{$}l<{$}} % math-mode version of lrc column types
\newcolumntype{R}{>{$}r<{$}} 
\newcolumntype{C}{>{$}c<{$}}

% Use more muted colors for links
% \hypersetup{colorlinks=True, linkcolor=blue!50!black, citecolor=black,
%   urlcolor=blue!50!black}
\hypersetup{hidelinks=True}

% Tweaks to siunitx configuration
\sisetup{
  % explicit""+" is useful for velocities
  retain-explicit-plus = true,
  % prefer 10^6 over 1 x 10^6
  retain-unity-mantissa = false,
  % Use x +/- e instead of x(e)  
  separate-uncertainty = true,
  % Make sure to pick up bold font when used in section heading for instance
  detect-weight = true,
}

\usepackage{upgreek}
%%%%%%%%%%%%%%%%%%%%%%%%%%%%%%%%%%%%%%%%%%%%%%%%%%
\usepackage{placeins}
%%%%% AUTHORS - PLACE YOUR OWN COMMANDS HERE %%%%%

% Title of the paper, and the short title which is used in the headers.
% Keep the title short and informative.
\title[Turbulence diagnostic trade-offs]{Trade-offs between noise and resolution in turbulence diagnostics}

% The list of authors, and the short list which is used in the headers.
% If you need two or more lines of authors, add an extra line using \newauthor
\author[J. García-Vázquez et al.]{
  J. García-Vázquez\textsuperscript{1}\thanks{jgarciav1600@alumno.ipn.mx},
  William J. Henney\textsuperscript{2}\thanks{w.henney@irya.unam.mx},
  and S. Jane Arthur\textsuperscript{2}
  \\
  \textsuperscript{1}\foreignlanguage{spanish}{%
    Escuela Superior de Física y Matemáticas, Instituto Politécnico Nacional, U.P. Adolfo López Mateos, Zacatenco, 07738 Ciudad de México, México}\\
  \textsuperscript{2}\foreignlanguage{spanish}{%
    Instituto de Radioastronomía y
    Astrofísica, Universidad Nacional Autónoma de México, Apartado
    Postal 3-72, 58090 Morelia, Michoacán, Mexico}\\
}

% These dates will be filled out by the publisher
\date{Accepted XXX. Received YYY; in original form ZZZ}

% Enter the current year, for the copyright statements etc.
\pubyear{2023}

% Don't change these lines
\begin{document}
\label{firstpage}
\pagerange{\pageref{firstpage}--\pageref{lastpage}}
\maketitle

% Abstract of the paper
\begin{abstract}
  % Context
  Turbulence plays a critical role in the evolution of H~II regions, yet recovering its statistical properties from observations remains challenging, especially under low signal-to-noise conditions.
  % What we did
  We use high-resolution integral field spectroscopic observations from MUSE/VLT of the Orion Nebula (M~42) to recover the turbulent velocity field across multiple emission lines.
  % How we did it
  We analyze the centroid velocity maps by computing the second-order structure function of the velocity fluctuations in the plane-of-sky. We apply a masking procedure based on surface brightness and systematically test binning to optimize the trade-off between instrumental noise and spatial resolution. A model is then fitted to the observed structure functions to extract the key turbulent parameters, including the correlation length $r_0$, slope $m$, and noise level.
  % What we found
  Masking and downsampling significantly improve the reliability and efficiency of structure function analysis. The fitted model successfully recovers the turbulent parameters for all binning levels, with higher bin sizes yielding better agreement with the "true" (noise-free, seeing-free) structure function. Differences in large-scale behavior among emission lines are observed, revealing line-specific turbulence signatures. Our method allows consistent recovery of turbulent parameters even under varying observational conditions. Comparison with KPNO echelle yields...
\end{abstract}

% Select between one and six entries from the list of approved keywords.
% Don't make up new ones.
\begin{keywords}
HII regions -- ISM: kinematics and dynamics -- turbulence 
\end{keywords}


%%%%%%%%%%%%%%%%%%%%%%%%%%%%%%%%%%%%%%%%%%%%%%%%%%

%%%%%%%%%%%%%%%%% BODY OF PAPER %%%%%%%%%%%%%%%%%%

\section{Introduction}

%Intro first para (what we are studying)
The process of star formation begins with the gravitational collapse of molecular clouds, followed by fragmentation into dense cores where stars form in clusters. 
Massive O- and B-type stars emerge within these stellar groups, and their intense ionizing radiation and stellar winds inject substantial energy and momentum into the surrounding gas, profoundly shaping the interstellar medium (ISM).

Observations of both molecular and ionized gas reveal complex velocity structures that deviate from simple motions. 
Instead, these motions exhibit characteristics consistent with turbulence. 
In this context, H~II regions—especially nearby ones—offer ideal laboratories for studying turbulent dynamics in the ionized ISM.

Turbulence in H~II regions is typically first characterized through the amplitude of velocity fluctuations. This can be studied using velocity dispersion along the line-of-sight (LOS) or through the variance of centroid velocities projected onto the plane-of-sky (POS). 
Statistical tools, such as the second-order structure function, enable the recovery of turbulence signatures in the POS and allow comparisons with theoretical models like Kolmogorov’s theory of incompressible turbulence.

A major challenge in analyzing turbulence from observational data lies in recovering the intrinsic turbulent behavior while minimizing the effects of instrumental noise, seeing, and resolution limits.
This becomes especially important in regions of low surface brightness or when high-resolution data are undersampled.

The Orion Nebula (M~42), located at a distance of 440 pc \citetext{\SI{1}{\arcsecond} = \SI{0.002}{pc}; \citealp{2008AJ....136.1566O}}, is the nearest massive star-forming region and one of the most studied H~II regions. 
Ionized by the O7~V star $\theta^1$~Ori~C, the nebula features complex physical and kinematic structures, with a rich population of young stars, stellar outflows, and Herbig-Haro objects. Its proximity and brightness make it a prime target for turbulence studies using spectroscopic observations.

Several works have analyzed turbulence in the Orion Nebula using various statistical methods \citep{von1951methode, munch1958internal, castaneda1988, 1992ApJ...387..229O, 2016MNRAS.455.4057M, arthur2016turbulence, garciav23}. 
However...

%Intro second para (how we are studying it)
In this study, we use integral field spectroscopic data from MUSE/VLT to analyze the turbulent velocity field of the Orion Nebula across multiple emission lines. 

% Intro para Description of our methodology
By combining a systematic masking approach and downsampling strategy, we aim to recover the true second-order spatial structure function of the velocity fluctuations and reliably extract the underlying turbulent parameters. 
This method enables inter-line and inter-dataset comparisons, providing insight into the structure and nature of turbulence in ionized gas.

\section{Methods}


\section*{Acknowledgements}

\section*{Data availability statement}
\label{sec:data-avail-stat}
All data and accompanying analysis programs used in this paper are available
from the github repository \url{https://github.com/JavGVastro/PhD.Paper}.

\bibliographystyle{mnras}
\bibliography{turb-refs}

%\clearpage


%
%%%%%%%%%%%%%%%%%%%%%%%%%%%%%%%%%%%%%%%%%%%%%%%%%%
%%%%%%%%%%%%%%%%%%%%% END %%%%%%%%%%%%%%%%%%%%%%%%
%%%%%%%%%%%%%%%%%%%%%%%%%%%%%%%%%%%%%%%%%%%%%%%%%%
% Don't change these lines
\bsp	% typesetting comment
\label{lastpage}
\end{document}

%%% Local Variables:
%%% mode: LaTeX
%%% TeX-master: t
%%% End:
